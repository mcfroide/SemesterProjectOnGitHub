%%%%%%%%%%%%%%%%%%%%%%%%%%%%%%%%%%%%%%%%%
% Beamer Presentation
% LaTeX Template
% Version 1.0 (10/11/12)
%
% This template has been downloaded from:
% http://www.LaTeXTemplates.com
%
% License:
% CC BY-NC-SA 3.0 (http://creativecommons.org/licenses/by-nc-sa/3.0/)
%
%%%%%%%%%%%%%%%%%%%%%%%%%%%%%%%%%%%%%%%%%

%----------------------------------------------------------------------------------------
%	PACKAGES AND THEMES
%----------------------------------------------------------------------------------------

\documentclass{beamer}

\mode<presentation> {

% The Beamer class comes with a number of default slide themes
% which change the colors and layouts of slides. Below this is a list
% of all the themes, uncomment each in turn to see what they look like.

%\usetheme{default}
%\usetheme{AnnArbor}
%\usetheme{Antibes}
%\usetheme{Bergen}
%\usetheme{Berkeley}
%\usetheme{Berlin}
%\usetheme{Boadilla}
\usetheme{CambridgeUS}
%\usetheme{Copenhagen}
%\usetheme{Darmstadt}
%\usetheme{Dresden}
%\usetheme{Frankfurt}
%\usetheme{Goettingen}
%\usetheme{Hannover}
%\usetheme{Ilmenau}
%\usetheme{JuanLesPins}
%\usetheme{Luebeck}
%\usetheme{Madrid}
%\usetheme{Malmoe}
%\usetheme{Marburg}
%\usetheme{Montpellier}
%\usetheme{PaloAlto}
%\usetheme{Pittsburgh}
%\usetheme{Rochester}
%\usetheme{Singapore}
%\usetheme{Szeged}
%\usetheme{Warsaw}

% As well as themes, the Beamer class has a number of color themes
% for any slide theme. Uncomment each of these in turn to see how it
% changes the colors of your current slide theme.

%\usecolortheme{albatross}
%\usecolortheme{beaver}
%\usecolortheme{beetle}
%\usecolortheme{crane}
%\usecolortheme{dolphin}
%\usecolortheme{dove}
%\usecolortheme{fly}
%\usecolortheme{lily}
%\usecolortheme{orchid}
%\usecolortheme{rose}
%\usecolortheme{seagull}
%\usecolortheme{seahorse}
%\usecolortheme{whale}
%\usecolortheme{wolverine}

%\setbeamertemplate{footline} % To remove the footer line in all slides uncomment this line
%\setbeamertemplate{footline}[page number] % To replace the footer line in all slides with a simple slide count uncomment this line

\setbeamertemplate{navigation symbols}{} % To remove the navigation symbols from the bottom of all slides uncomment this line
}

\usepackage{graphicx} % Allows including images
\usepackage{booktabs} % Allows the use of \toprule, \midrule and \bottomrule in tables
\usepackage{multimedia}
\usepackage{hyperref}
\usepackage{dsfont}

%----------------------------------------------------------------------------------------
%	TITLE PAGE
%----------------------------------------------------------------------------------------

\title[Low-Rank Tensor Completion]{Low-Rank 3D Tensor Completion} % The short title appears at the bottom of every slide, the full title is only on the title page

\author{Marine Froidevaux} % Your name
%Supervisors: Dr. J. Ballani, Prof. D. Kressner
\institute[EPFL] % Your institution as it will appear on the bottom of every slide, may be shorthand to save space
{
\'Ecole Polytechnique F\'ed\'erale de Lausanne \\ % Your institution for the title page
\medskip
\textit{marine.froidevaux@epfl.ch} % Your email address
}
\date{\today} % Date, can be changed to a custom date

\begin{document}

\begin{frame}
\titlepage % Print the title page as the first slide
\end{frame}

%\begin{frame}
%\frametitle{Overview} % Table of contents slide, comment this block out to remove it
%\tableofcontents % Throughout your presentation, if you choose to use \section{} and \subsection{} commands, these will automatically be printed on this slide as an overview of your presentation
%\end{frame}

%----------------------------------------------------------------------------------------
%	PRESENTATION SLIDES
%----------------------------------------------------------------------------------------

%------------------------------------------------
\section{Motivations} % Sections can be created in order to organize your presentation into discrete blocks, all sections and subsections are automatically printed in the table of contents as an overview of the talk
%------------------------------------------------

%\subsection{Subsection Example} % A subsection can be created just before a set of slides with a common theme to further break down your presentation into chunks

\begin{frame}
\frametitle{Motivations}
\begin{itemize}
\item Movies
\item Signal processing
\item Multiple parameters approximation
\end{itemize}
\end{frame}

\section{Overview of algebraic tools} % Sections can be created in order to organize your presentation into discrete blocks, all sections and subsections are automatically printed in the table of contents as an overview of the talk
%------------------------------------------------

%\subsection{Subsection Example} % A subsection can be created just before a set of slides with a common theme to further break down your presentation into chunks

\begin{frame}
\frametitle{Norm and matricization}
\begin{itemize}
\item Generalization of Frobenius norm for $\mathcal{X}\in \mathbb{R}^{I_1\times I_2 \times I_3}$:\\ \begin{center} $\| \mathcal{X} \|=\sqrt{\sum\limits_{i=1}^{I_1}\sum\limits_{j=1}^{I_2}\sum\limits_{k=3}^{I_3}x_{i,j,k}^2}$\end{center}
\item Unfolding $ \mathcal{X}$ in the $i^{th}$ dimension:\\ \begin{center} $X_{(i)}\in \mathbb{R}^{I_i\times (I_j\cdot I_k)}$, $j\neq i,$ $k\neq i,j$  \end{center}
\item $n$-mode product:\\ \begin{center} $\mathcal{Y}=\mathcal{X}\times_n U \iff Y_{(n)}=UX_{(n)}$  \end{center}
%\item rank of a tensor p464
\item $n$-rank of a tensor: \\ \begin{center} $rank_n(\mathcal{X}):=$ rank of $X_{(n)}$
%$\mathcal{X}$ is rank-$[R_1, R_2, R_3]$ if $R_n=rank_n(\mathcal{X})$ for $i=1,2,3$.
%$[R_1, R_2, R_3] $
\end{center}
\end{itemize} 
\end{frame}

\begin{frame}
\frametitle{Tucker Decomposition}
\begin{itemize}
\item Core tensor $\mathcal{G}\in \mathbb{R}^{R_1\times R_2\times R_3}$, principal components in each mode $A_1\in \mathbb{R}^{I_1\times R_1},A_2\in \mathbb{R}^{I_2\times R_2}, A_3\in \mathbb{R}^{I_3\times R_3}$:\\ \begin{center} $ \mathcal{X} =  \mathcal{G} \times_1 A_1\times_2 A_2 \times_3 A_3$\end{center}
\item Higher-Order SVD: Find a rank-$[R_1,R_2,R_3]$ core tensor and corresponding components that best capture the variation in mode-$n$, independently of other modes%:\\
%\begin{minipage}{0.48 \linewidth}
%\centering
%\begin{figure}\includegraphics[scale=0.5]{HOSVD.pdf} \end{figure}
%\end{minipage}
%\begin{minipage}{0.48 \linewidth}
%\centering
%\begin{figure}\includegraphics[scale=0.7]{AlgoHOSVD.pdf} \end{figure}
%\end{minipage}
\end{itemize}
\end{frame}

\begin{frame}[fragile] % Need to use the fragile option when verbatim is used in the slide
\frametitle{Higher Order Singular Value Decomposition}
\begin{block}{}
\begin{figure}
\includegraphics[width=0.6\linewidth]{AlgoHOSVD}
\end{figure}
\end{block}
\centering
\begin{figure}\includegraphics[scale=0.6]{HOSVD.pdf} \end{figure}
\end{frame}

\section{Description of the problem and algorithms} 
\begin{frame}
\frametitle{Movie reconstruction and Image Inpainting}
%MB and stuff
\begin{itemize}
\item Each frame is divided into Macro-Blocks (MBs) of size $N$ 
\item For each incomplete Macro-Block $P^0$, look for $K-1$ complete MBs minimizing\\
\begin{center} $S_i=\| P^0_{\Omega}-P^i_{\Omega}\|$ \end{center}
\item Stack all selected $P^i$ into a tensor $\mathcal{X}$ with $P^0$ at the top.
\item Ideally $\mathcal{X}=\mathcal{X}_l+\mathcal{E}=P\times_3\mathds{1}+\mathcal{E}$\\
\item Tucker decomposition: $\mathcal{X}_l=\mathcal{G}\times_1 A_1\times_2 A_2\times_3 A_3$\\
$\Rightarrow P=\mathcal{G}\times_1 A_1\times_2 A_2$ and $A_3=\mathds{1}$
\item Find $\mathcal{X}_l$ a $[R_1,R_2,R_3]$-rank approximation of $\mathcal{X}$, with $R_3$ small.
\item Two algorithms: HOSVD with Alternating Least Square and GeomCG
\end{itemize}

%Inpaiting

%Two Algorithms
\end{frame}

\subsection{ALS}

\begin{frame}[fragile] % Need to use the fragile option when verbatim is used in the slide
%\frametitle{Alternating Least Squares Algorithm}
\begin{block}{Alternating Least Square Algorithm}
%\begin{verbatim}
\begin{figure}
\includegraphics[width=0.7\linewidth]{AlgoALS}
\end{figure}
%\end{verbatim}
\end{block}
\end{frame}

\subsection{GeomCG}

\begin{frame}
\frametitle{GeomCG}
Optimization on the manifold
\end{frame}


\section{Examples}

\begin{frame}
\frametitle{Movie reconstruction}
Macro-Block size: N=8
%\begin{minipage}{0.48 \linewidth}
%\movie[externalviewer]{\includegraphics[scale=0.3]{mycorruptedposter.png}}{corruptedBus.avi}
%\hyperlinkmovie[poster]{corruptedBus.avi}{}
%\end{minipage}
%\begin{minipage}{0.48 \linewidth}
%\movie[externalviewer]{\includegraphics[scale=0.3]{myrecoveredposter.png}}{recoveredBus.avi}
%\hyperlinkmovie[poster]{recoveredBus.avi}{}
%\end{minipage}
\begin{center}
\movie[externalviewer]{\includegraphics[scale=0.5]{myposter.png}}{recoveredBusN8.avi}
\hyperlinkmovie[poster]{recoveredBusN8.avi}{}
\end{center}
\end{frame}

\begin{frame}
%\frametitle{Error Plot}
Relative error = $\frac{\| F_{original}-F_{restored} \|}{\|F_{original} \|}$
\begin{figure}
\includegraphics[width=0.7\linewidth]{pEffectPlot}
\end{figure}
\end{frame}

\begin{frame}
\frametitle{Importance of initial guess}
\begin{minipage}{0.48 \linewidth}
\centering
Educated guess
\begin{figure}
\includegraphics[width=1.1\linewidth]{XInitConvergence}
\end{figure}
\end{minipage}
\begin{minipage}{0.48 \linewidth}
\centering
Random guess
\begin{figure}
\includegraphics[width=1.1\linewidth]{XInitRandom}
\end{figure}
\end{minipage}
\end{frame}

\begin{frame}
\frametitle{Inpainting}
%\begin{minipage}{0.48 \linewidth}
\begin{figure}
\includegraphics[width=1\linewidth]{InpaintingFacade.png}\footnote{\tiny{http://www.briqueterie-chimot.fr/wpcproduct/la-brique-rouge-chimot/}}
\end{figure}
\begin{figure}
\includegraphics[width=1\linewidth]{InpaintingSnow.png}\footnote{\tiny{http://images.forwallpaper.com/files/images/0/0249/0249d4f1/113975/winter-park-snow-bench.jpg}}
\end{figure}
%\end{minipage}
%\begin{minipage}{0.48 \linewidth}
%\begin{figure}
%\includegraphics[width=1\linewidth]{InpaintingFacadeN6.png}
%\end{figure}
%\end{minipage}
\end{frame}

\begin{frame}
\frametitle{Comparison with GeomCG}
Compare results for rank 1
\end{frame}

\section*{References}
\begin{frame}
%\frametitle{}


\footnotesize{
\begin{thebibliography}{99} % Beamer does not support BibTeX so references must be inserted manually as below
\bibitem[ALS]{} D.T. Nguyen, M.D. Dao, T.D. Tran, \textit{The John Hopkins University}, 2011
\newblock Error Concealment Via 3-Mode Tensor Approximation
\newblock \emph{18th IEEE Conference on Image Processing}

\bibitem[GeomCG]{} D.Kressner, M. Steinlechner, B.Vandereycken, \textit{\'Ecole Polytehcnique F\'ed\'erale de Lausanne}, 2013
\newblock Low-Rank Tensor Completion by Riemannian Optimization

\bibitem[Kolda-Bader]{} T.G. Kolda, B.W. Bader, \textit{Sandia National Laboratories}, 2009
\newblock Tensor Decomposition and Applications
\newblock \emph{SIAM Review}, Vol.51, No.3, pp. 455-500
\end{thebibliography}
}
\end{frame}





%
%\begin{frame}
%\frametitle{Paragraphs of Text}
%Sed iaculis dapibus gravida. Morbi sed tortor erat, nec interdum arcu. Sed id lorem lectus. Quisque viverra augue id sem ornare non aliquam nibh tristique. Aenean in ligula nisl. Nulla sed tellus ipsum. Donec vestibulum ligula non lorem vulputate fermentum accumsan neque mollis.\\~\\
%
%Sed diam enim, sagittis nec condimentum sit amet, ullamcorper sit amet libero. Aliquam vel dui orci, a porta odio. Nullam id suscipit ipsum. Aenean lobortis commodo sem, ut commodo leo gravida vitae. Pellentesque vehicula ante iaculis arcu pretium rutrum eget sit amet purus. Integer ornare nulla quis neque ultrices lobortis. Vestibulum ultrices tincidunt libero, quis commodo erat ullamcorper id.
%\end{frame}
%
%%------------------------------------------------
%
%\begin{frame}
%\frametitle{Bullet Points}
%\begin{itemize}
%\item Lorem ipsum dolor sit amet, consectetur adipiscing elit
%\item Aliquam blandit faucibus nisi, sit amet dapibus enim tempus eu
%\item Nulla commodo, erat quis gravida posuere, elit lacus lobortis est, quis porttitor odio mauris at libero
%\item Nam cursus est eget velit posuere pellentesque
%\item Vestibulum faucibus velit a augue condimentum quis convallis nulla gravida
%\end{itemize}
%\end{frame}
%
%%------------------------------------------------
%
%\begin{frame}
%\frametitle{Blocks of Highlighted Text}
%\begin{block}{Block 1}
%Lorem ipsum dolor sit amet, consectetur adipiscing elit. Integer lectus nisl, ultricies in feugiat rutrum, porttitor sit amet augue. Aliquam ut tortor mauris. Sed volutpat ante purus, quis accumsan dolor.
%\end{block}
%
%\begin{block}{Block 2}
%Pellentesque sed tellus purus. Class aptent taciti sociosqu ad litora torquent per conubia nostra, per inceptos himenaeos. Vestibulum quis magna at risus dictum tempor eu vitae velit.
%\end{block}
%
%\begin{block}{Block 3}
%Suspendisse tincidunt sagittis gravida. Curabitur condimentum, enim sed venenatis rutrum, ipsum neque consectetur orci, sed blandit justo nisi ac lacus.
%\end{block}
%\end{frame}
%
%%------------------------------------------------
%
%\begin{frame}
%\frametitle{Multiple Columns}
%\begin{columns}[c] % The "c" option specifies centered vertical alignment while the "t" option is used for top vertical alignment
%
%\column{.45\textwidth} % Left column and width
%\textbf{Heading}
%\begin{enumerate}
%\item Statement
%\item Explanation
%\item Example
%\end{enumerate}
%
%\column{.5\textwidth} % Right column and width
%Lorem ipsum dolor sit amet, consectetur adipiscing elit. Integer lectus nisl, ultricies in feugiat rutrum, porttitor sit amet augue. Aliquam ut tortor mauris. Sed volutpat ante purus, quis accumsan dolor.
%
%\end{columns}
%\end{frame}
%
%%------------------------------------------------
%\section{Second Section}
%%------------------------------------------------
%
%\begin{frame}
%\frametitle{Table}
%\begin{table}
%\begin{tabular}{l l l}
%\toprule
%\textbf{Treatments} & \textbf{Response 1} & \textbf{Response 2}\\
%\midrule
%Treatment 1 & 0.0003262 & 0.562 \\
%Treatment 2 & 0.0015681 & 0.910 \\
%Treatment 3 & 0.0009271 & 0.296 \\
%\bottomrule
%\end{tabular}
%\caption{Table caption}
%\end{table}
%\end{frame}
%
%%------------------------------------------------
%
%\begin{frame}
%\frametitle{Theorem}
%\begin{theorem}[Mass--energy equivalence]
%$E = mc^2$
%\end{theorem}
%\end{frame}
%
%%------------------------------------------------
%
%\begin{frame}[fragile] % Need to use the fragile option when verbatim is used in the slide
%\frametitle{Verbatim}
%\begin{example}[Theorem Slide Code]
%\begin{verbatim}
%\begin{frame}
%\frametitle{Theorem}
%\begin{theorem}[Mass--energy equivalence]
%$E = mc^2$
%\end{theorem}
%\end{frame}\end{verbatim}
%\end{example}
%\end{frame}
%
%%------------------------------------------------
%
%\begin{frame}
%\frametitle{Figure}
%Uncomment the code on this slide to include your own image from the same directory as the template .TeX file.
%%\begin{figure}
%%\includegraphics[width=0.8\linewidth]{test}
%%\end{figure}
%\end{frame}
%
%%------------------------------------------------
%
%\begin{frame}[fragile] % Need to use the fragile option when verbatim is used in the slide
%\frametitle{Citation}
%An example of the \verb|\cite| command to cite within the presentation:\\~
%
%This statement requires citation \cite{p1}.
%\end{frame}
%
%%------------------------------------------------
%
%\begin{frame}
%\frametitle{References}
%\footnotesize{
%\begin{thebibliography}{99} % Beamer does not support BibTeX so references must be inserted manually as below
%\bibitem[Smith, 2012]{p1} John Smith (2012)
%\newblock Title of the publication
%\newblock \emph{Journal Name} 12(3), 45 -- 678.
%\end{thebibliography}
%}
%\end{frame}
%
%%------------------------------------------------
%
%\begin{frame}
%\Huge{\centerline{The End}}
%\end{frame}
%
%%----------------------------------------------------------------------------------------
%
\end{document} 