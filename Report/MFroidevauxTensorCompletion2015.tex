\documentclass[a4paper, 11pt]{article} % rajouter l'option [parskip] pour supprimer l'indentation et ajouter un saut de ligne à chaque début de paragraphe (halfparskip pour une demi-ligne)
\usepackage[utf8]{inputenc}
%\usepackage[latin1]{inputenc}
%\usepackage[applemac]{inputenc} % Pour avoir droit aux accents en compilant Mac Os Roman
%\usepackage{ucs}
\usepackage[english]{babel} % Pour franciser toutes les commandes automatiques (y compris les césures)
\usepackage[T1]{fontenc} % Définit les lettres accentuées comme caractères simples, et permet notamment d'éviter certains problèmes de césures 
\usepackage{lmodern} % Optimise encore l'utilisation des caractères accentués en utilisant l'encodage de police Latin Modern
\usepackage{textcomp} % Donne accès à certains symboles comme celui de l'euro par exemple
%\usepackage{amsmath} %Utile pour certaines pinailleries mathématiques. Attention à ne pas le charger si on n'en a pas besoin car il peut modifier certains environnements comme equnarray par exemple.
%\usepackage{eurosym}
\usepackage{graphics,graphicx} 	%pour inclure des graphiques
%\usepackage{subfigure}
%\usepackage{lscape}
%\usepackage{textcomp}
%\usepackage{wrapfig}		%figure dans le texte
%\usepackage{floatflt}		%figures dans le texte
%\usepackage{picins}		%figures dans le texte

\usepackage[colorlinks,bookmarks=false,linkcolor=blue,urlcolor=blue]{hyperref} %Pour les liens internet, hypertexte, email
\usepackage{geometry}		%pour les dimensions de la page
\usepackage{fancyhdr}		%entête et pied de page personnalisés, pour plus d'info voir ``entête latex'' dans favoris
\usepackage{amsmath,amssymb} %Utile pour certains symboles mathématiques
%\setlength{\parindent}{0cm} % Pour supprimer l'indentation en début de paragraphe
%\setlength{\parskip}{\baselineskip} % Pour programmer un saut de ligne avant chaque paragraphe, solution à éviter
\usepackage{lastpage}		%pour indiquer le nombre de pages
\usepackage{multirow}		%tableau sur plusieurs lignes
\usepackage{color}		
\usepackage{url}
%\usepackage{easybmat} Pour dessiner des matrices par blocs
\usepackage{subcaption}
\usepackage{pgfplots}


%Pour inclure du code C++
\usepackage{listings}
\usepackage{xcolor}
\usepackage{amssymb}
\usepackage{animate}

% Pour le pseudo-code
\usepackage{algorithm}
\usepackage[noend]{algpseudocode}



%modification de la mise en page
\geometry{a4paper,tmargin=33mm,bmargin=33mm,lmargin=28mm,rmargin=28mm}
\setlength{\parskip}{\medskipamount} 	%saut moyen entre §
\setlength{\parindent}{0pt}		%pas de retrait
%\pagestyle{headings}        		% Pour mettre des entêtes avec les titres des sections en haut de page
\pagestyle{fancy}
\lhead{ \textsc{}} \chead{} \rhead{Marine Froidevaux} \lfoot{\today}\cfoot{\thepage} \rfoot{} \renewcommand{\headrulewidth}{0.4pt} \renewcommand{\footrulewidth}{0.4pt}


%%initialisation pour le C++
%\lstset{language=C++, tabsize=2}
%\lstset{basicstyle=\small\ttfamily,
%keywordstyle=\color{blue}\bfseries,
%commentstyle=\color{red},
%stringstyle=\color{orange}
%showstringspaces=wrong}


%Initialisation for Matlab
\lstset{language=Matlab, keywords={break,case,catch,continue,else,elseif,end,for,function,
global,if,otherwise,persistent,return,switch,try,while},
basicstyle=\ttfamily,
keywordstyle=\color{blue},
commentstyle=\color{red},
stringstyle=\color{magenta},
numbers=left, numberstyle=\tiny\color{gray},
stepnumber=1,
numbersep=10pt,
backgroundcolor=\color{white},
tabsize=3,
showspaces=false,
showstringspaces=false, frame=single}



\def \be {\begin{equation*}}
\def \ee {\end{equation*}}
\def \bee {\begin{equation}}
\def \eee {\end{equation}}
\def \bm {\begin{pmatrix}}
\def \em {\end{pmatrix}}
\def \bd {\begin{displaymath} \left[ \begin{array}}
\def \ed {\end{array} \right. \end{displaymath}}
\def \ba {\begin{displaymath} \begin{array}}
\def \ea {\end{array} \end{displaymath}}
\def \bv {\begin{verbatim}}
\def \ev {\end{verbatim}}
\def \d {\partial}
\def \i {\infty}
\def \O {\mathcal{O}}
\def \X {\mathcal{X}}
\def \R {\mathbb{R}}
\def \t {\times}
\def \rank{\mbox{rank}}
\def \Y {\mathcal{Y}}
\def \C {\mathcal{C}}
\def \Mr {\mathcal{M}_{r}}


%\renewcommand{\thesubsubsection}{ \alph{subsubsection})}
%\renewcommand{\thesubsection}{}


\newcount\dotcnt\newdimen\deltay
\def\Ddot#1#2(#3,#4,#5,#6){\deltay=#6\setbox1=\hbox to0pt{\smash{\dotcnt=1
\kern#3\loop\raise\dotcnt\deltay\hbox to0pt{\hss#2}\kern#5\ifnum\dotcnt<#1
\advance\dotcnt 1\repeat}\hss}\setbox2=\vtop{\box1}\ht2=#4\box2}

%\makeatletter
%\renewcommand\thesubsection{}
%\renewcommand\thesubsubsection{\@arabic\c@section.\@arabic\c@subsection. \alph{subsubsection})}
%\makeatother

\makeatletter
\def\BState{\State\hskip-\ALG@thistlm}
\makeatother


\newcommand{\mail}[1]{{\href{mailto:#1}{#1}}}
\newcommand{\ftplink}[1]{{\href{ftp://#1}{#1}}}

\begin{document}
\title{3D Tensor Completion\\ \textsc{}}
%\author{Lukas König et Marine Froidevaux\\ \mail{lukasgeorg.koenig@epfl.ch} \\ \mail{marine.froidevaux@epfl.ch}}
\author{Marine Froidevaux, Supervisors: Dr. Jonas Ballani and Prof. Daniel Kressner }% \\ \mail{marine.froidevaux@epfl.ch}}

%\date{2011} % Si on ne met pas du tout la commande "date" le maketitle ajoute automatiquement la date du jour à la page de titre. Pour ne pas voir apparaître la date il faut écrire la commande vide \date{}.



\maketitle
%\begin{figure}[h!]
%\centering
%\includegraphics[scale=0.5]{electrode}
%\end{figure}
%
%\newpage
\thispagestyle{empty}
\pagenumbering{roman}
\setcounter{tocdepth}{2} % Pour afficher les paragraphes (profondeur 4) dans la table des matières
\tableofcontents
\newpage
\pagenumbering{arabic}



\section{Introduction}
Error concealment is a domain of image processing which deals with

Error typically in form of patches 

This report presents an error concealment algorithm, called Alternating Least Square, that reconstructs the corrupted patches by exploiting the low-rank property of a stack of similar patches. Results of the application of this algorithm to movie reconstruction and inpainting are presented here as well. 

Moreover, for comparison purpose, the same movies and images were reconstructed using the algorithm GeomCG for low-rank approximation.

\section{Basics of tensor algebra}
Before looking at the error concealment algorithm in details, a few basic tools of tensor algebra need to be defined.
This section is based on the paper by Kolda and Bader \cite{Kolda-Bader}, which should be referred back to for further details.

It is sufficient for our purpose to consider three-dimensional tensors only, but all algebraic concepts presented in this section can be generalised to higher dimensions.

Let $\X \in \R^{I_1\t I_2 \t I_3}$.
The \underline{i$^{th}$ matricisation} of $\X$ is a reordering of its entries into a matrix $X_{(i)} \in \R^{ I_i\t \prod_{j\neq i}I_j}$ such that the indices of $\X$ in the i$^{th}$ dimension become the row indices of $X_{(i)}$ and the indices of $\X$ in the other dimensions are rearranged in lexicographical order to become column indices of $X_{i}$. 

The $i-$rank of $\X$ is defined as the rank of its i$^{th}$ matricisation. The \underline{multilinear rank} of $\X$ is a 3-tuple grouping the $i$-rank of $\X$ in each of the three dimensions:
\be
\rank(\X):=\left[\rank(X_{(1)}),  \rank(X_{(2)}), \rank(X_{(3)})\right]
\ee

The \underline{i$^{th}$-mode produc}t of $\X$ with a matrix $M \in \R^{m\t I_i}$ is a tensor $\mathcal{Y} \in \R^{J_1\t J_2\t J_3}$, with $J_j=I_j$ for $j\neq i$, and $J_i=m$. $\mathcal{Y}$ is defined by means of the matricisation as follows:
\be
\mathcal{Y}=\X \t_{i} M \iff Y_{(i)}=MX_{(i)}
\ee
Note that the matricisation operation is a bijection from $ \R^{I_1\t I_2 \t I_3}$ into $\R^{ I_i\t \prod_{j\neq i}I_j}$. The product $\mathcal{Y}$ is thus uniquely determined by any of its matricisation.

The \underline{inner product} of two tensors $\X, \Y in \R^{I_1\t I_2 \t I_3}$ is the sum of the product of their entries, that is to say:
\be
<\X,\Y>:=\sqrt{\sum\limits_{i_1=1}^{I_1}\sum\limits_{i_2=1}^{I_2}\sum\limits_{i_3=1}^{I_3} x_{i_1,i_2,i_3}y_{i_1,i_2,i_3}}
\ee 
The induced \underline{norm} $\|\X\|:=\sqrt{<\X, \X>}$ is thus a generalisation of the matrix Frobenius norm to 3 dimensions.

Any tensor $\X$ of multilinear rank $\rank(\X)=[r_1, r_2, r_3]$ can be decomposed into a core tensor $\C \in \R^{r_1\t r_2 \t r_3}$ and three orthogonal basis matrices $U_i \in \R^{I_i\t r_i}$, such that
\be
\X=\C \t_1 U_1 \t_2 U_2 \t_3 U_3 
\ee 
This is called the \underline{Tucker Decomposition} and is the basis of the {High-Order Singular Value Decomposition} (HOSVD) algorithm. 

HOSVD aims at computing a rank-$\left[R_1,R_2, R_3\right]$ approximation of a tensor $\X$ of higher rank in the Tucker format. This is done by matricising $\X$ in each dimension $i=1,2,3$ and using the $R_i$ principal left singular vectors as basis $U_i$ of the Tucker decomposition. The pseudo-code for the HOSVD procedure is given in Algorithm \autoref{HOSVD}.

\begin{algorithm}
\caption{Higher-Order Sinngular Value Decomposition}\label{HOSVD}
\begin{algorithmic}[1]
\Procedure{HOSVD ($\X, R_1, R_2, R_3$)}{}
%\State for $ i=1,2,3$
\For {$i=1,2,3}$
\State $U_i \gets R_i$ leading left singular vectors of $X_{(i)}$ 
\EndFor
\State $\mathcal{G} \gets \X \t_1 U_1^\top \t_2 U_2^\top \t_3 U_3^\top$
\State \Return {$\mathcal{G}, U_1, U_2, U_3$}
%\BState \emph{top}:
%\If {$i > \textit{stringlen}$} \Return false
%\EndIf
%\State $j \gets \textit{patlen}$
%\BState \emph{loop}:
%\If {$\textit{string}(i) = \textit{path}(j)$}
%\State $j \gets j-1$.
%\State $i \gets i-1$.
%\State \textbf{goto} \emph{loop}.
%\State \textbf{close};
%\EndIf
%\State $i \gets i+\max(\textit{delta}_1(\textit{string}(i)),\textit{delta}_2(j))$.
%\State \textbf{goto} \emph{top}.
\EndProcedure
\end{algorithmic}
\end{algorithm}


This procedure can be viewed as a generalisation of the matrix Singular Value Decomposition (SVD) as a truncated SVD is performed on each matricisation of $\X$ to obtain a lower rank approximation. However, in contrast to the 2-dimensional case, the HOSVD does not lead to the best rank-$\left[R_1,R_2, R_3\right]$ approximation of $\X$. Let us denote the subspace of  all 3-dimensional tensors of rank $r=\left[R_1, R_2, R_3\right]$  by  $\Mr:=\{ \mathcal{T} \in \R^{I_1\t I_2 \t I_3} | \rank(\mathcal{T})=\left[R_1,R_2, R_3\right] \}$ and let $P_{\Mr}\X$ be a best approximation of $\X$ in the subspace $\Mr$ relatively to the norm defined above. If $P_{HO}\X$ represents the result of the HOSVD procedure applied to $\X$ then the following results holds \cite{GeomCG}:
\be
\|\X-P_{HO}\X \|\leqslant \sqrt{3} \| \X-P_{\Mr}\X \|
\ee

This means that a best low-rank approximation of $\X$  is only better by a factor $\frac{1}{\sqrt{3}}$ than what is computed by means of the HOSVD procedure.

\section{Selection of samples}


\section{Alternating Least Squares Algorithm (ALS)}

\section{GeomCG}
\section{Results}
\subsection{Bus Sequence}
\subsection{Inpainting}

\section{Conclusion}
\section{Suggestion for further research}

\begin{thebibliography}{9}
%\bibitem{Allan} 
%Allan Nielsen. 
%\textit{Solving the shallow water wave equation on HPC hardware. MATH454 Final Project}. 
%2015.

\bibitem{ALS} 
D.T. Nguyen, M.D. Dao and T.D. Tran. {The John Hopkins University}, 2011.
\textit{Error Concealment Via 3-Mode Tensor Approximation.} 18th IEEE Conference on Image Processing


\bibitem{GeomCG} 
D.Kressner, M. Steinlechner and B.Vandereycken.{ \'Ecole Polytechnique F\'ed\'erale de Lausanne}, 2013. \textit{Low-Rank Tensor Completion by Riemannian Optimization}


\bibitem{Kolda-Bader}  T.G. Kolda and B.W. Bader. {Sandia National Laboratories}, 2009.
\textit{Tensor Decomposition and Applications}. SIAM Review, Vol.51, No.3, pp. 455-500
\end{thebibliography}

%\begin{thebibliography}{99} % Beamer does not support BibTeX so references must be inserted manually as below
%\bibitem[ALS]{} D.T. Nguyen, M.D. Dao, T.D. Tran, \textit{The John Hopkins University}, 2011
%\newblock Error Concealment Via 3-Mode Tensor Approximation
%\newblock \emph{18th IEEE Conference on Image Processing}
%
%\bibitem[GeomCG]{} D.Kressner, M. Steinlechner, B.Vandereycken, \textit{\'Ecole Polytechnique F\'ed\'erale de Lausanne}, 2013
%\newblock Low-Rank Tensor Completion by Riemannian Optimization
%
%\bibitem[Kolda-Bader]{} T.G. Kolda, B.W. Bader, \textit{Sandia National Laboratories}, 2009
%\newblock Tensor Decomposition and Applications
%\newblock \emph{SIAM Review}, Vol.51, No.3, pp. 455-500
%\end{thebibliography}
%

\end{document}